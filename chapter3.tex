\chapter{Data Analytics} 
\section{Data Warehouse} 
\begin{description}
\item[Data Warehouse] subject-oriented, integrated, time-variant, and nonvolatile collection of data in support of management's decision making process
\item[Data Warehousing] The process of constructing and using data warehouses
\item[Query-Driven] When a query comes, wrappers translate the query for each DB. Integrators combine results from different DB.
\item[Update-Drive] Information from heterogenous sources is integrated in advance and stored in a DW for direct querying and analysis. High performance, but no most recent information. 
\end{description}

\section{On-line Analytical Processing}
\begin{description}
\item[OLAP] uses a multidimensional array representation.
\item[Data cube] is a multidimensional representation of data, together with all possible aggregates.
\item[Aggregates] mean the result by selecting a proper subset of the dimensions and summing over all the remaining dimensions.
\end{description} \par \noindent
Categories of Data Cube Measures
\begin{itemize}
\item Distributive: sum, count, min, max
\item Algebraic: average, std, maxN, minN
\item Holistic: median, mostFrequent, rank
\end{itemize}